%% Choose Different types of output.
%%%%%%%%%%%%%%%%%%%%%%%%%%%%%%%%%%%%%%%%%%%%%%%%%%%%%%%%%%%%%%%%%%%%%%%%%%%
%% \documentclass[notes=only,handout]{beamer}
%% \documentclass[notes]{beamer}
%% \documentclass[draft]{beamer}
\documentclass{beamer}
%% \includeonlyframes{walk_in,debrief}
%%%%%%%%%%%%%%%%%%%%%%%%%%%%%%%%%%%%%%%%%%%%%%%%%%%%%%%%%%%%%%%%%%%%%%%%%%

%% Load necessary packages.
%%%%%%%%%%%%%%%%%%%%%%%%%%%%%%%%%%%%%%%%%%%%%%%%%%%%%%%%%%%%%%%%%%%%%%%%%%%
%% mdframed and framed are both used to create frames that could span multiple
%% pages.
\usepackage[framemethod=tikz]{mdframed}

%% fontawesome offers a number of high quality common icon.
\usepackage{fontawesome}

%% Table lines.
\usepackage{booktabs}
\newcommand{\head}[1]{\textbf{#1}}

%% Write verbatim text with math escaped.
\usepackage{listings}
\lstset{
  basicstyle=\ttfamily,
  mathescape
}

%% For citation.
%% \usepackage[backend=bibtex, style=numeric]{biblatex}
\usepackage[backend=biber, style=authortitle]{biblatex}
\bibliography{citations}
%%%%%%%%%%%%%%%%%%%%%%%%%%%%%%%%%%%%%%%%%%%%%%%%%%%%%%%%%%%%%%%%%%%%%%%%%%%

%% Font.
%%%%%%%%%%%%%%%%%%%%%%%%%%%%%%%%%%%%%%%%%%%%%%%%%%%%%%%%%%%%%%%%%%%%%%%%%%%
%% \setmainfont{Times New Roman}
%%%%%%%%%%%%%%%%%%%%%%%%%%%%%%%%%%%%%%%%%%%%%%%%%%%%%%%%%%%%%%%%%%%%%%%%%%%

%% Load an outertheme.
%%%%%%%%%%%%%%%%%%%%%%%%%%%%%%%%%%%%%%%%%%%%%%%%%%%%%%%%%%%%%%%%%%%%%%%%%%%
\useoutertheme{miniframes}


%% Define an color theme.
%% Customized dark color theme.
%%%%%%%%%%%%%%%%%%%%%%%%%%%%%%%%%%%%%%%%%%%%%%%%%%%%%%%%%%%%%%%%%%%%%%%%%%%
\mode<presentation>

\setbeamercolor{normal text}{fg=white, bg=black!90}
\setbeamercolor{note page}{fg=black, bg=white!90}
\setbeamercolor{structure}{fg=white}

\setbeamercolor{alerted text}{fg=green}

\setbeamercolor{item projected}{use=item, fg=black, bg=item.fg!35}

\setbeamercolor*{palette primary}{use=structure, fg=structure.fg}
\setbeamercolor*{palette secondary}{use=structure, fg=structure.fg!95!black}
\setbeamercolor*{palette tertiary}{use=structure, fg=structure.fg!90!black}
\setbeamercolor*{palette quaternary}{use=structure, fg=structure.fg!95!black, bg=black!80}

\setbeamercolor*{framesubtitle}{fg=white}

\setbeamercolor*{block title}{parent=structure, bg=black!60}
\setbeamercolor*{block body}{fg=black, bg=black!10}
\setbeamercolor*{block title alerted}{parent=alerted text, bg=black!15}
\setbeamercolor*{block title example}{parent=example text, bg=black!15}

\mode<all>
%%%%%%%%%%%%%%%%%%%%%%%%%%%%%%%%%%%%%%%%%%%%%%%%%%%%%%%%%%%%%%%%%%%%%%%%%%%

%% Style definition for mdframed.
%%%%%%%%%%%%%%%%%%%%%%%%%%%%%%%%%%%%%%%%%%%%%%%%%%%%%%%%%%%%%%%%%%%%%%%%%%%
\def\hmargin{-30}
\def\hinnermargin{20pt}
\def\vsmallmargin{10pt}

%% Note that the foreground and background colors are set separately from
%% beamer theme.
\mdfdefinestyle{description}{
  tikzsetting={draw=black, fill=black, fill opacity=0.7, line width=0pt},
  backgroundcolor=none,
  fontcolor=white,
  leftmargin=\hmargin,
  rightmargin=\hmargin,
  innerleftmargin=\hinnermargin,
  innerrightmargin=\hinnermargin,
  innertopmargin=\vsmallmargin,
  innerbottommargin=\vsmallmargin,
  skipbelow=\baselineskip,
  skipabove=\baselineskip
  %% roundcorner=4pt
  %% leftline=false,
  %% rightline=false
}

\def\vbigmargin{0.5in}

\mdfdefinestyle{heading}{
  tikzsetting={draw=black, fill=black, fill opacity=0.7, line width=0pt},
  backgroundcolor=none,
  fontcolor=white,
  leftmargin=\hmargin,
  rightmargin=\hmargin,
  innertopmargin=\vbigmargin,
  innerbottommargin=\vbigmargin,
  skipbelow=\baselineskip,
  skipabove=\baselineskip
  %% roundcorner=4pt
  %% leftline=false,
  %% rightline=false
}

%%%%%%%%%%%%%%%%%%%%%%%%%%%%%%%%%%%%%%%%%%%%%%%%%%%%%%%%%%%%%%%%%%%%%%%%%%%

%% Environment and command utilities.
%%%%%%%%%%%%%%%%%%%%%%%%%%%%%%%%%%%%%%%%%%%%%%%%%%%%%%%%%%%%%%%%%%%%%%%%%%%
\newenvironment{headingframe}{
  \begin{mdframed}[style=heading]
    \begin{center}
}{
    \end{center}
  \end{mdframed}
}

\newcommand{\stresstext}[1]{
    {\Huge \textsc{#1}}
}
%%%%%%%%%%%%%%%%%%%%%%%%%%%%%%%%%%%%%%%%%%%%%%%%%%%%%%%%%%%%%%%%%%%%%%%%%%%

%% %%%%%%%%%%%%%%%%%%%%%%%%%%%%%%%%%%%%%%%%%%%%%%%%%%%%%%%%%%%%%%%%%%%%%%%%
%%
%% WHERE THE CONTENT BEGINS
%%
%% %%%%%%%%%%%%%%%%%%%%%%%%%%%%%%%%%%%%%%%%%%%%%%%%%%%%%%%%%%%%%%%%%%%%%%%%

%% Section for title, author, institute etc.
%%%%%%%%%%%%%%%%%%%%%%%%%%%%%%%%%%%%%%%%%%%%%%%%%%%%%%%%%%%%%%%%%%%%%%%%%%%
\title[Short Title]{Long Title}
%% \subtitle[Subtitle]{Subtitle}
\author{Name}
\date{}
%% \date{\today}
%% \institute{Chinese University of Hong Kong}

%%%%%%%%%%%%%%%%%%%%%%%%%%%%%%%%%%%%%%%%%%%%%%%%%%%%%%%%%%%%%%%%%%%%%%%%%%%
\begin{document}


\section*{Walking In}
%% Cover frame.
%%%%%%%%%%%%%%%%%%%%%%%%%%%%%%%%%%%%%%%%%%%%%%%%%%%%%%%%%%%%%%%%%%%%%%%%%%%
%% Uncomment the usebackgroundtemplate at the beginning and end of the frame to
%% use image as background.
%% {\usebackgroundtemplate{\includegraphics[height=1.0\paperheight]{./image/Solvay_conference_1927.jpg}}
    \begin{frame}[plain, label=walk_in]
      \vspace{5mm}
      \begin{mdframed}[style=description]
          \begin{center}
            {\Huge \inserttitle}\linebreak
            \vspace{5mm}

            {\large \textsc{\insertauthor}}\linebreak
            \insertinstitute\linebreak
            \insertdate\linebreak
          \end{center}
          \begin{flushright}
            {\footnotesize
              \href{mailto: lishuai918@gmail.com}{{\faEnvelope}  lishuai918@gmail.com}
            }
          \end{flushright}
      \end{mdframed}
      %% \titlepage{}
    \end{frame}
%% }
\note{}

%%%%%%%%%%%%%%%%%%%%%%%%%%%%%%%%%%%%%%%%%%%%%%%%%%%%%%%%%%%%%%%%%%%%%%%%%%%

\section*{Outline}
%%%%%%%%%%%%%%%%%%%%%%%%%%%%%%%%%%%%%%%%%%%%%%%%%%%%%%%%%%%%%%%%%%%%%%%%%%%
\frame{\tableofcontents[pausesections]}

\AtBeginSubsection[]
{
  \begin{frame}<beamer>[allowframebreaks]
    \frametitle{Table Of Contents}
    \tableofcontents[currentsection,currentsubsection]
  \end{frame}
}
%%%%%%%%%%%%%%%%%%%%%%%%%%%%%%%%%%%%%%%%%%%%%%%%%%%%%%%%%%%%%%%%%%%%%%%%%%%

\begin{frame}
  \frametitle{Tabular Example}

  \begin{tabular}{|p{0.2\linewidth}|p{0.6\linewidth}|}
    COL ONE & COL TWO \pause \\
    COL ONE & COL TWO        \\
  \end{tabular}
\end{frame}

\begin{frame}
  \frametitle{Automatically In-place Itemize Example}

  \begin{itemize}[<+-| only@+>]
  \item ONE
  \item TWO
  \end{itemize}
\end{frame}

\begin{frame}
  \frametitle{Verbatim Math}

  \begin{lstlisting}
                      computation
    neural systems  $\leftrightarrow$  behavior
  \end{lstlisting}
\end{frame}


%% {\usebackgroundtemplate{\includegraphics[height=1.0\paperheight]{./image/cute.jpg}}
\begin{frame}[label=thank_you]
  \vspace{-1.5in}
  \begin{mdframed}[style=description]
    {\large Thank you.}
  \end{mdframed}
\end{frame}
%% }

\end{document}
%%%%%%%%%%%%%%%%%%%%%%%%%%%%%%%%%%%%%%%%%%%%%%%%%%%%%%%%%%%%%%%%%%%%%%%%%%%

%%% Local Variables:
%%% mode: latex
%%% TeX-master: t
%%% End:
