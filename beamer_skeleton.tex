%% Choose Different types of output.
%%%%%%%%%%%%%%%%%%%%%%%%%%%%%%%%%%%%%%%%%%%%%%%%%%%%%%%%%%%%%%%%%%%%%%%%%%%
%% \documentclass[handout]{beamer}
\documentclass{beamer}
%%%%%%%%%%%%%%%%%%%%%%%%%%%%%%%%%%%%%%%%%%%%%%%%%%%%%%%%%%%%%%%%%%%%%%%%%%%

%% Load necessary packages.
%%%%%%%%%%%%%%%%%%%%%%%%%%%%%%%%%%%%%%%%%%%%%%%%%%%%%%%%%%%%%%%%%%%%%%%%%%%
%% mdframed and framed are both used to create frames that could span multiple
%% pages.
\usepackage[framemethod=tikz]{mdframed}
\usepackage{framed}

%% fontawesome offers a number of high quality common icon.
\usepackage{fontawesome}
%% fontspec deals with font for xelatex, though most of packages load it by
%% default.
\usepackage{fontspec}

%% xcolor deals with color and offers \colorbox.
\usepackage{xcolor}
%%%%%%%%%%%%%%%%%%%%%%%%%%%%%%%%%%%%%%%%%%%%%%%%%%%%%%%%%%%%%%%%%%%%%%%%%%%

% Load a theme.
%%%%%%%%%%%%%%%%%%%%%%%%%%%%%%%%%%%%%%%%%%%%%%%%%%%%%%%%%%%%%%%%%%%%%%%%%%%
%% \usetheme{Berkeley}
%% \usetheme{Frankfurt}
%% \usecolortheme{seahorse}
%% \usecolortheme{rose}
%% \usetheme{Warsaw}
%% \usepackage{beamerthemesplit}

%% \usetheme{Warsaw}
%%%%%%%%%%%%%%%%%%%%%%%%%%%%%%%%%%%%%%%%%%%%%%%%%%%%%%%%%%%%%%%%%%%%%%%%%%%


%% Define an color theme.
%% Customized dark color theme.
%%%%%%%%%%%%%%%%%%%%%%%%%%%%%%%%%%%%%%%%%%%%%%%%%%%%%%%%%%%%%%%%%%%%%%%%%%%
\mode<presentation>

\setbeamercolor{normal text}{fg=white, bg=black!90}
\setbeamercolor{structure}{fg=white}

\setbeamercolor{alerted text}{fg=red!85!black}

\setbeamercolor{item projected}{use=item, fg=black, bg=item.fg!35}

\setbeamercolor*{palette primary}{use=structure, fg=structure.fg}
\setbeamercolor*{palette secondary}{use=structure, fg=structure.fg!95!black}
\setbeamercolor*{palette tertiary}{use=structure, fg=structure.fg!90!black}
\setbeamercolor*{palette quaternary}{use=structure, fg=structure.fg!95!black, bg=black!80}

\setbeamercolor*{framesubtitle}{fg=white}

\setbeamercolor*{block title}{parent=structure, bg=black!60}
\setbeamercolor*{block body}{fg=black, bg=black!10}
\setbeamercolor*{block title alerted}{parent=alerted text, bg=black!15}
\setbeamercolor*{block title example}{parent=example text, bg=black!15}

\mode<all>
%%%%%%%%%%%%%%%%%%%%%%%%%%%%%%%%%%%%%%%%%%%%%%%%%%%%%%%%%%%%%%%%%%%%%%%%%%%


%%%%%%%%%%%%%%%%%%%%%%%%%%%%%%%%%%%%%%%%%%%%%%%%%%%%%%%%%%%%%%%%%%%%%%%%%%%
\title[Title on the footer]{Title}
%% \subtitle[Subtitle]{Subtitle}
\author{Shuai Li}
\date{}
%% \date{\today}
%% \institute{Chinese University of Hong Kong}

\def\hmargin{0}

\newmdenv[
  tikzsetting={draw=white, fill=black, fill opacity=0.7, line width=0pt},
  backgroundcolor=none,
  leftmargin=\hmargin,
  rightmargin=\hmargin,
  innertopmargin=4pt,
  skipbelow=\baselineskip,
  skipabove=\baselineskip,
  %% roundcorner=4pt
  leftline=false,
  rightline=false
]{TitleBox}

%%%%%%%%%%%%%%%%%%%%%%%%%%%%%%%%%%%%%%%%%%%%%%%%%%%%%%%%%%%%%%%%%%%%%%%%%%%
\begin{document}


\begin{frame}[plain]
  \vspace{5mm}
  \begin{TitleBox}
    {\usebeamercolor[fg]{title}
    \begin{center}
      {\LARGE \inserttitle}\linebreak
      \vspace{5mm}

      {\large \insertauthor}\linebreak
      \insertinstitute\linebreak
      \insertdate\linebreak
    \end{center}
    \begin{flushright}
      {\footnotesize
        %% \href{http://www.falsters.net/daniel}{{\FA \faHome} www.falsters.net/daniel}
        \href{mailto: lishuai918@gmail.com}{{\faEnvelope}  lishuai918@gmail.com}
      }
    \end{flushright}
  }
  \end{TitleBox}
  %% \titlepage{}
\end{frame}

\section*{Outline}
\frame{\tableofcontents[pausesections]}

\AtBeginSubsection[]
{
  \begin{frame}<beamer>
    \frametitle{Layout}
    \tableofcontents[currentsection,currentsubsection]
  \end{frame}
}

\section{Command Test}

\begin{frame}
  \frametitle{box test}
  %% Box will not be broken when exceeding line width.
  \mbox{mbox mbox mbox mbox mbox mbox mbox mbox mbox mbox mbox mbox mbox mbox mbox }
  \fbox{fbox fbox fbox fbox fbox fbox fbox fbox fbox fbox fbox fbox fbox fbox fbox }
  \framebox{fbox fbox fbox fbox fbox fbox fbox fbox fbox fbox fbox fbox fbox fbox fbox }
  \begin{framed}
    framed framed framed framed
  \end{framed}
  \begin{framed}
    framed framed framed framed framed framed framed framed framed framed framed framed
  \end{framed}
  \parbox{0.5\textwidth}{parbox parbox parbox parbox parbox }
\end{frame}

\begin{frame}
  \frametitle{box test (continued)}
  \fbox{
    \parbox[b][5em][b]{\paperwidth}{Paperwidth fbox using pbox}
  }
  \colorbox{red}{Color Box}
\end{frame}


\end{document}
%%%%%%%%%%%%%%%%%%%%%%%%%%%%%%%%%%%%%%%%%%%%%%%%%%%%%%%%%%%%%%%%%%%%%%%%%%%

%%% Local Variables:
%%% mode: latex
%%% TeX-master: t
%%% End:
