
%% %%%%%%%%%%%%%%%%%%%%%%%%%%%%%%%%%%%%%%%%%%%%%%%%%%%%%%%%%%%%%%%%%%%%%%%%
%%
%% WHERE THE CONTENT BEGINS
%%
%% %%%%%%%%%%%%%%%%%%%%%%%%%%%%%%%%%%%%%%%%%%%%%%%%%%%%%%%%%%%%%%%%%%%%%%%%

%% Section for title, author, institute etc.
%%%%%%%%%%%%%%%%%%%%%%%%%%%%%%%%%%%%%%%%%%%%%%%%%%%%%%%%%%%%%%%%%%%%%%%%%%%
\title[Short Title]{Long Title}
%% \subtitle[Subtitle]{Subtitle}
\author{Name}
\date{}
%% \date{\today}
%% \institute{Chinese University of Hong Kong}

%%%%%%%%%%%%%%%%%%%%%%%%%%%%%%%%%%%%%%%%%%%%%%%%%%%%%%%%%%%%%%%%%%%%%%%%%%%
\begin{document}


\section*{Walking In}
%% Cover frame.
%%%%%%%%%%%%%%%%%%%%%%%%%%%%%%%%%%%%%%%%%%%%%%%%%%%%%%%%%%%%%%%%%%%%%%%%%%%
%% Uncomment the usebackgroundtemplate at the beginning and end of the frame to
%% use image as background.
%% {\usebackgroundtemplate{\includegraphics[height=1.0\paperheight]{./image/Solvay_conference_1927.jpg}}
    \begin{frame}[plain, label=walk_in]
      \vspace{5mm}
      \begin{mdframed}[style=description]
          \begin{center}
            {\Huge \inserttitle}\linebreak
            \vspace{5mm}

            {\large \textsc{\insertauthor}}\linebreak
            \insertinstitute\linebreak
            \insertdate\linebreak
          \end{center}
          \begin{flushright}
            {\footnotesize
              \href{mailto: lishuai918@gmail.com}{{\faEnvelope}  lishuai918@gmail.com}
            }
          \end{flushright}
      \end{mdframed}
      %% \titlepage{}
    \end{frame}
%% }
\note{}

%%%%%%%%%%%%%%%%%%%%%%%%%%%%%%%%%%%%%%%%%%%%%%%%%%%%%%%%%%%%%%%%%%%%%%%%%%%

\section*{Outline}
%%%%%%%%%%%%%%%%%%%%%%%%%%%%%%%%%%%%%%%%%%%%%%%%%%%%%%%%%%%%%%%%%%%%%%%%%%%
\frame{\tableofcontents[pausesections,hideallsubsections]}

\AtBeginSubsection[]
{
  \begin{frame}<beamer>[allowframebreaks]
    \frametitle{Table Of Contents}
    \tableofcontents[currentsection,currentsubsection]
  \end{frame}
}
%%%%%%%%%%%%%%%%%%%%%%%%%%%%%%%%%%%%%%%%%%%%%%%%%%%%%%%%%%%%%%%%%%%%%%%%%%%

\begin{frame}
  \frametitle{Tabular Example}

  \begin{tabular}{|p{0.2\linewidth}|p{0.6\linewidth}|}
    COL ONE & COL TWO \pause \\
    COL ONE & COL TWO        \\
  \end{tabular}
\end{frame}

\begin{frame}
  \frametitle{Automatically In-place Itemize Example}

  \begin{itemize}[<+-| only@+>]
  \item ONE
  \item TWO
  \end{itemize}
\end{frame}

%% tells beamerthat the frame contents is “fragile.” This means that the frame
%% contains text that is not “interpreted as usual.” This case applies
%% to verbatim text, which is, obviously, interpreted somewhat differently from
%% normal text.
\begin{frame}[fragile]
  \frametitle{Verbatim Math}

  Hello World.
  \begin{lstlisting}
The world is yours.
  \end{lstlisting}
  Hello World.
\end{frame}


\begin{frame}
  \frametitle{Pop Up An Example}
  Some text.
  \pause

  \only<2>{
    \begin{textblock*}{64mm}(32mm,0.25\textheight)
      \begin{mdframed}[style=description]
        An awesome example.
        \begin{itemize}
        \item ONE
        \item TWO
        \end{itemize}
      \end{mdframed}
    \end{textblock*}
  }
\end{frame}

\begin{frame}
  \frametitle{Multi-Columns}
  % If one of the column of your text is too long, it may seems that the left
  % image or corresponding other contents will be dropped down. Normally, if
  % such case happen, it is mostly due to you arrange you content the wrong
  % way. However, if such situation does happen, you could add *T* option to
  % the columns environment, like {columns}[t,onlytextwidth].
  \begin{columns}[onlytextwidth]
    \begin{column}{0.5\textwidth}
      \centering
      \rule{100pt}{150pt}
    \end{column}
    \begin{column}{0.5\textwidth}
      Some text.
    \end{column}
  \end{columns}
\end{frame}


%% {\usebackgroundtemplate{\includegraphics[height=1.0\paperheight]{./image/cute.jpg}}
\begin{frame}[label=thank_you]
  \vspace{-1.5in}
  \begin{mdframed}[style=description]
    {\large Thank you.}
  \end{mdframed}
\end{frame}
%% }

%%% Local Variables:
%%% mode: latex
%%% TeX-master: "main"
%%% End:
